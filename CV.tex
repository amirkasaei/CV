\documentclass[]{CV}

\usepackage{amssymb}
\usepackage{outlines}
\usepackage{multicol}
\usepackage{hyperref}
%\usepackage{geometry}
%\geometry{a4paper, margin=1cm}


\fullname{SEYED AMIR KASAEI}

\begin{document}
\resumeheader
{\linkedin{amir-kasaei}}
{\email{a.kasaei@me.com}}
{\github{amirkasaei}}
{}


\vspace{-4mm}
\begin{flushright}
\website{amirkasaei.github.io}
\end{flushright}

\vspace{-7mm}
\begin{section}{EDUCATION}
\begin{itemize}
\item \textbf{Bachelor of Science, Computer Engineering} \hfill (Sep 2019 -- Aug 2023)\newline
{University of Guilan \newline GPA:3.98/4 (Grade: 19.58) - \textbf{Ranked 1st}}
\end{itemize}
\end{section}

%%%%%%%%%%%%%%%%%%%%%%%%%%%%%%%%%%%%%%%%%%%%%%%%%%%%%%%
\vspace{-2mm}
\begin{section}{HONORS \& AWARDS}
\begin{itemize}
\item \textbf{Ranked 1st} among B.Sc. Computer Engineering Students (\href{https://drive.google.com/file/d/1RfuN4xRnWvJ53gFADrlBrSxIEHqgPVWS/view?usp=sharing}{Certificate in Persian}) \hfill Sep 2019 -- Aug 2023 

\item M.Sc of Software Engineering \textbf{Admission} at \textbf{Sharif University of Technology} \hfill Sep 2023
\newline as an \textbf{Exceptional Talented Student} 

\item \textbf{Tuition Waiver}, Bachelor of Science, University of Guilan \hfill Sep 2019 -- Aug 2023 
\end{itemize}
\end{section}
%%%%%%%%%%%%%%%%%%%%%%%%%%%%%%%%%%%%%%%%%%%%%%%%%%%%%%%
\begin{section}{RESEARCH INTERESTS}
    \textbf{Machine Learning, Deep Learning, Data Mining, Computer Vision, Bioinformatics, Medical Image Processing, Natural Language Processing,  Meta-Learning}
\end{section}
\vspace{-2mm}
%%%%%%%%%%%%%%%%%%%%%%%%%%%%%%%%%%%%%%%%%%%%%%%%%%%%%%%
\begin{section}{RESEARCH PAPERS}
\begin{itemize}
\item \textbf{MAML on Human Sperm Abnormality Detection} \hfill (in progress)\newline
\textbf{Seyed Amir Kasaei}, Amir Mohammad  Ezzati, Amin Mokaddar Daemdoost, Seyed Abolghasem Mirroshandel.
\end{itemize}
\end{section}
\vspace{-2mm}

%%%%%%%%%%%%%%%%%%%%%%%%%%%%%%%%%%%%%%%%%%%%%%%%%%%%%%%
\begin{section}{EXPERIENCES}
\begin{multicols}{2}
\begin{itemize}
\item {\textbf{IT Support Technician} \newline Guilan Soha Seir \newline Sep 2019 - Present}
\item {\textbf{Teaching Assistant of Digital Circuits} \newline Univeristy of Guilan \newline Sep 2022 - Jan 2023 \newline Instructor: Dr. Mahhdi Aminian}
\item {\textbf{Teaching Assistant of Artificial Intelligence} \newline Univeristy of Guilan \newline Sep 2022 - Jan 2023 \newline Instructor: Dr. Yasaman Boreshban}
\item {\textbf{Teaching Assistant of Microelectronic Circuits} \newline Univeristy of Guilan \newline Feb 2023 - Jun 2023 \newline Instructor: Dr. Mahhdi Aminian}
\end{itemize}
\end{multicols}
\end{section}


%%%%%%%%%%%%%%%%%%%%%%%%%%%%%%%%%%%%%%%%%%%%%%%%%%%%%%%
\vspace{-4mm}
\begin{section}{SELECTED ACCOMPLISHED PROJECTS { - (Other Projects on \href{https://github.com/amirkasaei}{Github})}}
\begin{itemize}

\item \textbf{MAML on Human Sperm Abnormality Detection} - \href{https://github.com/amirkasaei/Modified-Human-Sperm-Morphology-Analysis}{Github repository}
\begin{itemize}
        \item We used MAML model to detected human sperm abnormality on its head, vacoule, acrosome, and tail. \vspace{-2mm}
        \item In this project we used \textbf{Learn2learn, Scikit-learn, PyTorch, Numpy, Matplotlib, seaborn}.
    \end{itemize}
    
\item \textbf{Image Segementation on CamVid using SegNet} - \href{https://github.com/amirkasaei/Image-Segementation-CamVid-using-SegNet}{Github repository}
   \begin{itemize}
        \item We implemented SegNet model to partition input images into 32-class regions .\vspace{-2mm}
        \item In this project we used \textbf{Pandas, Scikit-learn, Keras, Tensorflow, Numpy, Matplotlib}.
    \end{itemize}
    
     \item \textbf{Multi-Class Weather Classification} - \href{https://github.com/amirkasaei/Multi-Class-Weather-Classification}{Github repository}
   \begin{itemize}
        \item We implemented Simple CNN, Resnet And Inception Net for image classification.\vspace{-2mm}
        \item In this project we used \textbf{Pandas, Scikit-learn, Keras, Tensorflow, Numpy, Matplotlib}.
    \end{itemize}
    
\item \textbf{Arabic Broken Plurals} - \href{https://github.com/amirkasaei/Arabic-Broken-Plurals}{Github repository}
   \begin{itemize}
        \item We implemented a character based Machine Translation Model model to predict plural form of arabic words.\vspace{-2mm}
        \item In this project we used \textbf{BERT Embedding, Pandas, Scikit-learn, Keras, Tensorflow, Numpy}.
    \end{itemize}
\pagebreak  

\item \textbf{An Odd Music Generator} - \href{https://github.com/amirkasaei/An-Odd-Music-Generator}{Github repository}
   \begin{itemize}
        \item We implemented a Denoising Auto Encoder model to denois the input song, a Deep Neural Network to Recognize notes in the input song and a Language model to predict the next note of  the song.\vspace{-2mm}
        \item In this project we used \textbf{Scikit-learn, Keras, Tensorflow, Numpy, Librosa, Matplotlib}.
    \end{itemize} 
    
      
    
    
   \item \textbf{Multi-Lable Text Classification} - \href{https://github.com/amirkasaei/Multi-Lable-Text-Classification}{Github repository}
   \begin{itemize}
        \item We implemented a multi task language model to predict class type of subjects and also the sentimental analysis of sentences.\vspace{-2mm}
        \item In this project we used \textbf{Pandas, Scikit-learn, Keras, Tensorflow, Numpy, Hazm, NLTK, Matplotlib}.
    \end{itemize}

%\pagebreak

\end{itemize}
\end{section}

%%%%%%%%%%%%%%%%%%%%%%%%%%%%%%%%%%%%%%%%%%%%%%%%%%%%%%%
\begin{section}{NOTABLE COURSES}
\begin{multicols}{2}
    \begin{itemize}
	\item \textbf{Fundamental of Data Mining} - 20 / 20
	\item \textbf{Special Topics 1 (Deep Learning)} - 19.8 / 20
	\item \textbf{Special Topics 2 (Introduction to Machine Learning)} - 18.16 / 20
	\item \textbf{Fundamental of Speech and Natural Language Processing} - 20 / 20
	\item \textbf{Artificial Intelligence and Expert Systems} - 20 / 20
	
	\item \textbf{Principles of Database Design} - 20 / 20
	\item \textbf{Data Structures} - 20 / 20
	\item \textbf{Software Testing} - 20 / 20
	\item \textbf{Algorithm Design} - 20 / 20
	\item \textbf{Fundamentals of Compiler Design} - 20 / 20
	\item \textbf{The Theory of Formal Languages ​​and Automata} - 19.75 / 20
	
	\item \textbf{Engineering Statistics and Probability} - 20 / 20
	\item \textbf{Signal and Systems} - 19.5 / 20
	\item \textbf{Engineering Mathematics} - 19.75 / 20
	\item \textbf{Differential Equations} - 18.75 / 20
	\item \textbf{Discrete Mathematics} - 19.75 / 20
	
    \end{itemize}
    \end{multicols}
\end{section}

%%%%%%%%%%%%%%%%%%%%%%%%%%%%%%%%%%%%%%%%%%%%%%%%%%%%%%%
\vspace{-4mm}
\begin{section}{LICENSES \& CERTIFICATIONS}
\begin{multicols}{2}
\begin{itemize}
\item {\textbf{Deep Learning Specialization} \newline DeepLearning.AI \newline Instructor: Andrew Ng \newline Aug 2022 \newline \href{https://www.coursera.org/account/accomplishments/specialization/certificate/LGYU3FW9F9AX}{Credential}}

\item {\textbf{Machine Learning Specialization} \newline DeepLearning.AI \newline Instructor: Andrew Ng \newline Sep 2022 \newline \href{https://www.coursera.org/account/accomplishments/specialization/certificate/JFDSEZCH8ECY}{Credential}}

\item {\textbf{Fundamentals of Reinforcement Learning} \newline University of Alberta \newline Instructor: Martha and Adam White \newline Aug 2022 \newline \href{https://www.coursera.org/account/accomplishments/certificate/L2QGQYLFC8DL}{Credential}}
\item {\textbf{Crash Course on Python} \newline Google \newline Aug 2022  \newline \href{https://www.coursera.org/account/accomplishments/certificate/V9RF53KNW4TS}{Credential}}
\end{itemize}
\end{multicols}
\end{section}


%%%%%%%%%%%%%%%%%%%%%%%%%%%%%%%%%%%%%%%%%%%%%%%%%%%%%%%
\vspace{-4mm}
\begin{section}{SKILLS}
    \begin{itemize}
        \item \textbf{Programming Language :}~Python, Java, C++, VHDL
        \item \textbf{Data Visualization :}~Matplotlib, Numpy, Panadas, Seaborn
        \item \textbf{Web development :}~HTML, CSS, Bootstrap, jQuery, PHP,  MySQL
        \item \textbf{Machine Learning \& Deep Learning :} Scikit-learn, Keras, Tensorflow, Pytorch, Learn2learn, MAML
        \item \textbf{Computer Vision :}~CNN, Image Segmentation, Image Classification, Image Processing
        \item \textbf{Natural language processing :}~NLTK, hazm, Stemmer, Lemmatizer, Tokenizer, RNN,GRU, LSTM, BERT
        \item \textbf{Collaboration and Communication Tools : }~Slack, GitHub, Skype
    \end{itemize}
\end{section}

%%%%%%%%%%%%%%%%%%%%%%%%%%%%%%%%%%%%%%%%%%%%%%%%%%%%%%%
\pagebreak
\begin{section}{REFERENCES}
\begin{subsectionnobullet}{Dr. Seyed Abolghasem Mirroshandel}{mirroshandel@guilan.ac.ir}{University of Guilan}{\href{https://scholar.google.com/citations?user=WGH3eIsAAAAJ&hl=en}{Google Scholar}}
    \item {Associate Professor of Computer Engineering, University of Guilan, Rasht, Iran}
\end{subsectionnobullet}

\begin{subsectionnobullet}{Dr. Mahdi Aminian}{mahdi.aminian@guilan.ac.ir}{University of Guilan}{\href{https://scholar.google.com/citations?user=YVxXqIAAAAAJ&hl=en}{Google Scholar}}
\item {Assistant Professor of Computer Engineering, University of Guilan, Rasht, Iran}
\end{subsectionnobullet}

\begin{subsectionnobullet}{Dr. Farid Feyzi}{feizi@guilan.ac.ir}{University of Guilan}{\href{https://scholar.google.com/citations?user=HEBT11YAAAAJ&hl=en}{Google Scholar}}
\item {Assistant Professor of Computer Engineering, University of Guilan}
\end{subsectionnobullet}

\begin{subsectionnobullet}{Dr. Yasaman Boreshban}{yasaman.boreshban@sharif.edu}{University of Guilan}{\href{https://scholar.google.com/citations?hl=en&user=dKskDg8AAAAJ&hl=en}{Google Scholar}}
\item {Lecturer, Sharif University of Technology, University of Guilan}
\end{subsectionnobullet}

\begin{subsectionnobullet}{Dr. Aida Khozaee}{a.khozaee@ut.ac.ir}{University of Guilan}{\href{https://scholar.google.com/citations?user=eO3UXvEAAAAJ&hl=en}{Google Scholar}}
\item {Lecturer, University of Guilan}
\end{subsectionnobullet}

\end{section}

\end{document}